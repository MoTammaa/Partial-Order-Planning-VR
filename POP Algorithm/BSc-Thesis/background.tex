\chapter{Background}\label{chap:background}

\section{Introduction to Planning}
Planning in \acs{AI} is a fundamental aspect in the field that allows agents to formulate sequences of actions and strategies
to achieve a specific goal. It used in a wide range of fields where agents need to make decisions and take actions based on the
current state of the environment.

Classical planning is a type of planning that is used in \acs{AI} to solve problems that can be represented as a set of states and actions.
It deals with straightforward actions \& with predictable and deterministic environments, where the agent can predict the outcome of its actions.
The challenge in classical planning is to construct a sequence of actions that will transform the initial state
of the environment into a desired goal state, while dealing with exponential growth in the search space, and dealing with the
actions and steps in chronological order.

Among the different types of planning, we have \textbf{state space planning}, which is a type of planning that is used in \acs{AI} to solve
problems searching through a set of states and actions,
and \textbf{plan space planning}, which is a type of planning that is used in \acs{AI} to solve problems searching
through a set of plans and actions. In this thesis, we will focus on plan space planning, and more specifically on \acf{POP}.


\section{Partial Order Planning}
\acf{POP} is a plan-space search algorithm. Unlike other planning algorithms, \ac{POP} is partially ordered plan search.
This gives it the advantage over total order planning algorithms that it can use problem decomposition, work
on several subproblems in parallel independently, solve them with several subplans, and then merge the subplans into a final plan.


\ac{POP} uses least commitment strategy. It is a type of planning that does not require the planner to commit to a specific order of actions.
Instead, the planner can choose to leave some actions unordered, and the planner can choose to order actions only when necessary.
This allows the planner to explore a larger space of possible plans, and it allows the planner to find plans that
are more flexible and more robust. Partial Order Planning is a powerful and flexible planning algorithm that has
been used in a wide range of applications, including robotics, natural language processing, and automated planning.


\subsection{Formal Definition of Partial Plans}
A partial order plan is a tuple $\pi = (A, \prec, B, L)$ where:
\begin{itemize}
  \item $A$ is a set of actions, or partially instantiated Operators.
  \item $\prec$ is a set of ordering constraints between actions in the form of $a_i \prec a_j$
  \item $B$ is a set of bindings.
  \item $L$ is a set of causal links.
\end{itemize}
are the components of a partial plan.



\subsection{Consistency of Partial Plans}
A partial order plan $\pi = (A, \prec, B, L)$ is consistent if it satisfies the following conditions:
\begin{itemize}
  \item The transitive closure of the ordering constraints $\prec$ is a strict partial order.
  \item every subsitution $\sigma $ which binds a variable $x$ to a value in its allowed domain $D_x$ is consistent with the bindings in all other constraints in $B$.
\end{itemize}