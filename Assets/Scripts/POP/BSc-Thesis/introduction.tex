\chapter{Introduction}
\label{chap:intro}

\ac{AI} is a field that has been around for a long time. It is used in many fields such as medicine, finance, and transportation. In addition, it can be used to solve complex problems that are difficult for humans to solve. One of the existing problems in the classical \ac{AI} is the problem of planning. Planning is the process of finding a sequence of actions that will achieve a goal. It is an important problem in \ac{AI} because it is used in many applications such as robotics, scheduling, and game playing. There are many planning algorithms that have been developed over the years. One of the most popular planning algorithms is the \ac{POP} algorithm. The \ac{POP} algorithm is a powerful algorithm that can solve complex planning problems. Despite the importance of the algorithm, it can be difficult to learn.

Simulation is a powerful tool that can be used to study complex systems. It is used in many fields such as engineering, medicine, and finance. Simulation is used to study or model the behavior of a system. It is used to predict the behavior of a system under different conditions. \ac{VR} is a type of simulation that is used to create a virtual environment that is similar to the real world. It is used to create an immersive experience for the user. \ac{VR} is used in many applications such as training, gaming, and even military \cite{WikiVR}. One of the most popular applications of \ac{VR} is in the field of education. \ac{VR} is used to create virtual environments that are used to teach students about different subjects to be able to visualize and interact with the subject. Visualizing algorithms has been a challenge for many students. Thus, the idea of visualizing algorithms on software was hugely adopted through the years.

\section{Motivation}

Students in the field of \ac{AI} face many challenges when studying planning algorithms. One of the challenges is the complexity of the algorithms. Another challenge is the lack of visualization tools that can help students understand the algorithms. The lack of visualization tools makes it difficult for students to understand the algorithms like the \ac{POP} algorithm. As a result, implementing a \ac{VR} environment, that can visualize the \ac{POP} algorithm, can help students understand the algorithm better. This can help in grasping the concepts of the algorithm and gaining a better knowledge of how it works.

\section{Aim}

The aim of this project is to see the effect of using a \ac{VR} environment to visualize the \ac{POP} algorithm. The \ac{VR} environment will allow students to interact with the algorithm and see how it works. The \ac{VR} environment will provide a visual representation of the algorithm that will help students understand the algorithm better. The \ac{VR} environment will also provide a way for students to try and solve planning problems themselves using the \ac{POP} algorithm. Finally, we will evaluate the effectiveness of the \ac{VR} environment in helping students understand the \ac{POP} algorithm.

\section{Outline}
In the next chapter, we will discuss the background of the project, including the \ac{POP} algorithm, \ac{VR}, and Unity. In Chapter \ref{chap:design_and_implementation}: \textit{Design and Implementation}, we will discuss the design and implementation of the \ac{POP} algorithm engine and the \ac{VR} environment. Then, in Chapter \ref{chap:evaluation_results}: \textit{Evaluation Results}, we will discuss how we evaluated the \ac{VR} environment, the results of the evaluation, and the feedback we received from the participants. Finally, in Chapter \ref{chap:concl_future_work}: \textit{Conclusion and Future Work}, we will conclude the thesis and discuss the limitations of the project and the future work that can be done.


