\chapter{Conclusion and Future Work}\label{chap:concl}


In this thesis, we have presented a \ac{VR} environment that visualizes the \ac{POP} algorithm. The \ac{VR} environment provides a visual representation of the algorithm that aims to help students understand the algorithm better. The environment allows students to interact with the algorithm, see how it works, and try and solve planning problems themselves using the \ac{POP} algorithm. Unity 3D Engine was used to develop the virtual environment. The environment has been evaluated using a user study. The \ac{POP} algorithm was explained to the 20 participants, who were part of the user study. The participants were then asked to solve planning problems using the \ac{POP} algorithm. After that, the participants were asked to fill out a questionnaire to evaluate the effectiveness of the \ac{VR} environment in helping them understand the algorithm. The questionnaire was based on the \ac{SUS} questionnaire and some additional questions that were added to evaluate the effect of the \ac{VR} environment on the participants' understanding of the algorithm. The results of the user study show that the \ac{VR} environment is effective in helping students understand the \ac{POP} algorithm. The participants were able to solve planning problems using the \ac{POP} algorithm. The system also passed the \ac{SUS} test with a score of 79.125 which is considered above average. These results show that the users found the system easy to use and understand. The \ac{VR} environment for the \ac{POP} algorithm was proven successful in helping students understand the algorithm better.

\section{Limitations}

The \ac{VR} environment that was developed in this thesis has some limitations. The main limitation is that, in \ac{POP} algorithm engine, the only ways the planner can resolve threats are by using promotion and demotion. The planner does not support other the third way of resolving threats which is stricting the binding constraints in which we add a new binding constraint to render the effect of the threatening action non-unifiable with the threatened link condition. An approach was implemented and tested, but it was not successful.


\section{Future Work}

There are several directions for future work. Some possible future work includes improving the \ac{POP} algorithm engine, and others include improving the \ac{VR} environment. In addition, more user studies can be conducted to evaluate the effectiveness of the \ac{VR} environment.

\subsection{Improving the POP Algorithm Engine}

The \ac{POP} algorithm engine can be improved in multiple ways. First, the support for the third way of resolving threats by binding constraints can be added. This will make the planner more powerful and able to solve more complex planning problems. Second, some heuristics can be improved or added to the planner to help it find solutions faster. One of the most important heuristics that can be improved is the heuristic function used in the \ac{A*} search. The heuristic function can be improved to make it more accurate and admissible. Another heuristic that can be improved is the heuristic function used in choosing the \textit{agenda} pair to achieve. Finally, the planner execution can be parallelized to make it run faster.

\subsection{Improving the VR Environment}

The \ac{VR} environment can be improved in multiple ways. First, the environment can be extended to support more planning problems or planning algorithms. This will make the environment more useful for students who want to learn about different planning algorithms. Second, one of the feedbacks that we received from the participants is that the environment can be improved by adding a visual demonstration to guide the user in the \ac{VR}. This can help the user know the environment better and how to interact with it. Finally, the environment can be improved by adding more interactive elements to make it more engaging for the user. This can help the user stay focused and interested in the environment.

\subsection{Improving the User Study}

In the future, more user studies can be conducted to evaluate the effectiveness of the \ac{VR} environment. The user study can be extended to include more participants from different backgrounds. Additionally, future user studies can be extended to include a pre-test and post-test to evaluate the effect of the \ac{VR} environment on the participants' understanding of the algorithm. Moreover, the user study can be extended to include a control group to compare the effectiveness of the \ac{VR} environment with other visualization tools. Finally, this \ac{VR} environment can be used in a real running course to evaluate the effectiveness of the environment and compare the grades of the students in the previous years, who did not use the environment, with the grades of the students in the year, when the new \ac{VR} environment was used.
